%%
%% This is file `sample-sigconf.tex',
%% generated with the docstrip utility.
%%
%% The original source files were:
%%
%% samples.dtx  (with options: `sigconf')
%% 
%% IMPORTANT NOTICE:
%% 
%% For the copyright see the source file.
%% 
%% Any modified versions of this file must be renamed
%% with new filenames distinct from sample-sigconf.tex.
%% 
%% For distribution of the original source see the terms
%% for copying and modification in the file samples.dtx.
%% 
%% This generated file may be distributed as long as the
%% original source files, as listed above, are part of the
%% same distribution. (The sources need not necessarily be
%% in the same archive or directory.)
%%
%% The first command in your LaTeX source must be the \documentclass command.
\documentclass[sigconf]{acmart}

%%
%% \BibTeX command to typeset BibTeX logo in the docs
\AtBeginDocument{%
  \providecommand\BibTeX{{%
    \normalfont B\kern-0.5em{\scshape i\kern-0.25em b}\kern-0.8em\TeX}}}

%% Rights management information.  This information is sent to you
%% when you complete the rights form.  These commands have SAMPLE
%% values in them; it is your responsibility as an author to replace
%% the commands and values with those provided to you when you
%% complete the rights form.
\setcopyright{acmcopyright}
\copyrightyear{2020}
\acmYear{2020}
%\acmDOI{10.1145/1122445.1122456}

%% These commands are for a PROCEEDINGS abstract or paper.
\acmConference[CEP '20]{CEP '20: ACM Computing Education Practice}{January 9, 2020}{Durham, UK}
\acmBooktitle{CEP '20: Proceedings of the 3rd Conference on Computing Education Practice,
  Jan 9, 2020, Durham, UK}
\acmPrice{15.00}
%\acmISBN{978-1-4503-9999-9/18/06}


%%
%% Submission ID.
%% Use this when submitting an article to a sponsored event. You'll
%% receive a unique submission ID from the organizers
%% of the event, and this ID should be used as the parameter to this command.
%%\acmSubmissionID{123-A56-BU3}

%%
%% The majority of ACM publications use numbered citations and
%% references.  The command \citestyle{authoryear} switches to the
%% "author year" style.
%%
%% If you are preparing content for an event
%% sponsored by ACM SIGGRAPH, you must use the "author year" style of
%% citations and references.
%% Uncommenting
%% the next command will enable that style.
%%\citestyle{acmauthoryear}

%%
%% end of the preamble, start of the body of the document source.
\begin{document}

%%
%% The "title" command has an optional parameter,
%% allowing the author to define a "short title" to be used in page headers.
\title{Overcoming the Challenges of Teaching Cybersecurity in UK Computer Science
Degree Programmes}

%% 
%% The "author" command and its associated commands are used to define
%% the authors and their affiliations.
%% Of note is the shared affiliation of the first two authors, and the
%% "authornote" and "authornotemark" commands
%% used to denote shared contribution to the research.


\begin{comment}

\author{Tom Crick}
\affiliation{%
  \institution{Swansea University}
  \city{Swansea}
  \country{UK}
}
\email{thomas.crick@swansea.ac.uk}


\author{James H. Davenport}
\affiliation{%
  \institution{ University of Bath}
  \city{Bath}
  \country{UK}
}
\email{j.h.davenport@bath.ac.uk}

\author{Alastair Irons}
\affiliation{%
  \institution{ Sunderland University}
  \city{Sunderland}
  \country{UK}
}
\email{alastair.irons@sunderland.ac.uk}

\author{Tom Prickett}
\affiliation{%
  \institution{ Northumbria University}
  \city{Newcastle upon Tyne}
  \country{UK}
}
\email{tom.prickett@northumbria.ac.uk}
\end{comment}


%%
%% By default, the full list of authors will be used in the page
%% headers. Often, this list is too long, and will overlap
%% other information printed in the page headers. This command allows
%% the author to define a more concise list
%% of authors' names for this purpose.
\renewcommand{\shortauthors}{Crick, Davenport,  Irons, and Prickett.}
%%
%% The abstract is a short summary of the work to be presented in the
%% article.
\begin{abstract}
  An article published in the Harvard Business Review in August 2019
argued that ``{\emph{Every Computer Science Degree Should Require a
Course in Cybersecurity}}''\cite{cable_2019}; in the UK, universities
-- alongside government, industry and professional bodies -- have been
championing this over recent years, focusing on computer science and
cognate undergraduate degrees programmes. One professional body --
BCS, The Chartered Institute for IT -- has been mandating this in
accredited undergraduate degree programmes since
2015\cite{Cricketal2019}. Delivering cybersecurity effectively across
general computer science programmes presents a number of challenges
related to pedagogy, underpinning educational resources, available
skills and technical resources. This paper explores the progress to
date, as well as a starting call to arms to the UK higher education sector by
highlighting a number of future challenges and opportunities.
\end{abstract}

%%
%% The code below is generated by the tool at http://dl.acm.org/ccs.cfm
%% Please copy and paste the code instead of the example below.
%%
\begin{CCSXML}
<ccs2012>
<concept>
<concept_id>10002978</concept_id>
<concept_desc>Security and privacy</concept_desc>
<concept_significance>500</concept_significance>
</concept>
<concept>
<concept_id>10003456.10003457.10003527.10003529</concept_id>
<concept_desc>Social and professional topics~Accreditation</concept_desc>
<concept_significance>500</concept_significance>
</concept>
</ccs2012>
\end{CCSXML}

\ccsdesc[500]{Security and privacy}
\ccsdesc[500]{Social and professional topics~Accreditation}

% To do add cybersecurity to this

%% Keywords. The author(s) should pick words that accurately describe
%% the work being presented. Separate the keywords with commas.
\keywords{Accreditation, Cybersecurity, Computer Science Education}


%%
%% This command processes the author and affiliation and title
%% information and builds the first part of the formatted document.
\maketitle


\section {What is it?}
%A short description of the practice you're presenting

This paper explores the diversity of challenges relating to the
teaching of cybersecurity in UK higher education degree programmes,
from policy, through to pedagogy and practice. It frames these challenges through
concerns with the quality and availability of underpinning educational resources, the
competencies and skills of faculty (especially focusing on pedagogy
and assessment), and technical resources related to delivering sound
cybersecurity content in general computer science and cognate
degrees. There is a serious demand for cybersecurity specialists in the
Uk and globally (estimates vary, but are always large); there is
significant and growing higher education provision related to
specialist undergraduate and postgraduate courses focusing on varying
aspects of cybersecurity (for example cybersecurity, digital
forensics, ethical hacking, computer security, networks and security,
etc). To make our digital systems and products more secure, all in IT
need to know \emph{some} cybersecurity --- thus, there is a case for
depth as well as
breadth~\cite{manson+pike:2014,davenport-et-al:latice2016}. This is
not a new concern~\cite{Parr2014a}, but it is a growing one. Computer
science and cognate disciplines are evolving to meet these demands --
both in school-level education, as well as tertiary -- however,
doing so is not without challenges. This paper explores the progress
to date in the UK, highlights challenges for the future, as well as
identifying a number of potential enhancement activities for the domain.

\section{Why are you doing it?}	
%What happened before? What is it changing / replacing / improving? What gap is it filling?
\begin{comment}
\begin{quote}
	``{\emph{...[need to] change the culture in your organisation around cyber security; to try to do for cyber what has been done so successfully for health and safety, for example, over the last ten years --- to get everybody to take it seriously; to take the risk management process seriously and drive that down through the organisation.}}''\\
	\hfill Robert Hannigan~\cite{Hannigan2019a}, former Director of GCHQ
\end{quote}
\end{comment}

Cybersecurity is becoming increasing pivotal to the operation of organisations of all sizes and organisations are increasingly expected to make reasonable adjustments to protect their activities. The UK Government is encouraging organisations of all sizes and types to take this seriously \cite{Hannigan2019a}. 

\begin{comment}

This need to build knowledge, skills and capacity in the area of cybersecurity has also led to the establishment of a number of strategic policy initiatives from a number of national governments, for example the publication in 2016 of the UK's Cyber Security Strategy~\cite{ukcyberstrategy:2016} (along with the setting up of the National Cyber Security Centre, as well as increased scrutiny of the resilience of the UK's critical national infrastructure~\cite{lordscyberreport:2018}; also industry-focused initiatives such as Cyber Essentials~\cite{ncsc2017ce}, the EU Cybersecurity Act~\cite{eucyber2018} (which reinforces the mandate of the EU Agency for Cybersecurity: ENISA, the European Union Agency for Network and Information and Security), or the National Initiative for Cybersecurity Education (NICE) in the USA~\cite{NICE}.

\end{comment}

This focus on cybersecurity, includes calls for formal education -- school-level as well as tertiary -- to respond to this situation, at the individual level and via recommended curricula~\cite{mcgettrick-et-al:sigcse2014,ACM2017b} and professional accreditation requirements~\cite{BCS2018a,NCSC2017}. This is further reinforced by a wider focus on digital skills and computer science education reform, especially across the nations of the UK~\cite{brown-et-al:toce2014,murphy-et-al:programming2017,tryfonas+crick:petra2018,moller+crick:jce2018}. Embedding cybersecurity in computer science and related degrees is now the norm within the UK. %However it is one thing to teach cybersecurity and another thing to teach it well. 

A number of Professional, Statutory and Regulatory Bodies (PSRBs) have
responded to these expectations by adjusting their accreditation
requirements. In the UK, BCS, The Chartered Institute for IT (BCS) has
had a requirement to include information security in the curriculum
since 2010, and has expected coverage of an agreed minimum
cybersecurity syllabus since 2015, with the result that all accredited
universities should be compliant by 2020 (due to the five-year
accreditation cycle). More precisely, accredited degrees have been
expected to demonstrate coverage of ``{\emph{2.1.9 Knowledge and
    understanding of information security issues in relation to the
    design, development and the use of information systems}}''
\cite[p.~30]{BCS2018a} since 2010 with an enhanced cybersecurity
related definition of what this entails since 2015
\cite[p.~17--18]{BCS2018a}. However, it is one thing to teach
cybersecurity, but another to do it well. 
%This paper provides an evaluation of this.

\section {Where does it fit}

%A short description of your teaching context. You may, for instance, include a description of intake, class size, curriculum sequence; anything that's necessary for others to understand your situation. How do things work at your institution?

One way to evaluating how well cybersecurity is taught is to reflect upon the pedagogic approach used, the underpinning resources available, the disciplinary expertise of faculty and the required technical resources.

\subsection{What pedagogical approach to adopt?}
What is the most appropriate way to teach cybersecurity? \cite{Weiss:2013:THC:2527148.2527180} highlights there are benefits from teaching this in a practical manner. Real world case studies can be employed \cite{BritishAirways2018a}. Use can be made of guest lectures by industrialists to share practical insights and hence providing students with micro-exposure to the world of work is another positive contribution. One further approach is the inclusion of appropriate cybersecurity standards within the curricula.

The PCI DSS \cite{PCI2018b} is one such standard that has been used in precisely this manner. PCI DSS underpins all processing of credit/debit cards. Nevertheless, it is very rarely mentioned in generalist computer scientist courses. This would not matter so much if everyone handling payments data were sent by their employers on an effective PCI DSS course. However, the payments business is now so spread across websites, often run by small and medium enterprises (SME), or non-specialists. Even larger enterprises are not immune: \cite{BritishAirways2018a} reports that the recent British Airways breach was caused by a failure to adhere to PCI DSS in website maintenance.

Another way of adopting a more practical pedagogy is by teaching cybersecurity through the lens of hacking or the hacker curriculum \cite{bratus2010teaching}. Such an approach facilitates students to be more experimental and creative in their exploration of the discipline and can have corresponding benefits for their engagement. 

\begin{comment}


It is an interesting question as to whether standards such as PCI DSS should be addressed within degree courses (clearly degree courses can never cover all standards) or whether they should be addressed in professional training courses. However, the current situation is not ideal from the perspective of industry (or users of systems). The inclusion of key standards could be seen as a valuable enhancement activity to how cybersecurity it taught.

\end{comment}

% \subsection{How well do the under-pinning educational resources
% support the delivery of cybersecurity}
\subsection{Quality of resources to support cybersecurity education?}
Effective teaching requires appropriate supporting resources. The extent to which appropriate resources are available and suitable will be evaluated next. This evaluation highlights a number of occasions when underpinning resources could be improved.

\subsubsection{SQL Injection}
It is 15 years since \cite{Guimaraesetal2004} wrote ``{\emph{All the topics listed above should be presented in the first Database Course}}'', and the first such topic was SQL injection \cite{SPIDynamics2002,Anonymous2018b}. SQL injection as an attack has been around for twenty years \cite{HornerHyslip2017a}, has its own cartoon and website. Nevertheless SQL injection is still a major weakness: number one in the Open Web Application Security Project (OWASP) Top 10 \cite{OWASP2017a}, and has been in the Top 10 since at least 2003. 

%CyBOK states ``{\emph{\ldots a wide range of attack techniques for exploiting SQL injection or script injection are known and documented.}}''~\cite{Bristol2018a}.

Clearly such a major weakness should be well-taught; in general it is hard to determine what is actually delivered as part of a specific degree programme, but a reasonable proxy for this is the content of recommended textbooks. This was the rationale for a 2019 analysis of database textbooks used by 44 of the top 50 computer science departments in the USA~\cite{Drop2019}.  There were seven such books, but three books accounted for the 36 of the 44 universities. Five of the seven (30 of the 44) had no mention of SQL injection. Of the other two, the more popular one has a seriously flawed discussion, and the other, while generally excellent, had a presentational problem~\cite{Drop2019}.

\subsubsection{The Case of Java}
%A recommendation for future work is a comprehensive survey equivalent to \cite{Drop2019} for  Java textbooks. 
\begin{comment}
Took out footnotes for space

Many textbooks go nowhere near security applications.  But this means that the programmer who has to implement security is left to the documentation of the package/API being used, and to informal resources. \cite{Mengetal2018a} analysed 503 cybersecurity-related postings on the popular Stack Overflow online resource.  53\% were about the Spring Security framework  \footnote{\url{https://projects.spring.io/spring-security/}}, dominated by authentication (45\%). The discussion \cite[\S4.3.1]{Mengetal2018a} of cross-site request forgery (CSRF) is especially worrying.  By default, Spring implicitly enables protection against this. But all the accepted answers to CSRF-related failures simply suggested disabling the check. There were no negative comments about this, and indeed a typical response is 
\end{comment}

Many textbooks go nowhere near security applications, despite their ubiquity.  But this means that the programmer who has to implement security is left to the documentation of the package/API being used, and to informal resources. \cite{Mengetal2018a} analysed 503 cybersecurity-related postings on the popular Stack Overflow online resource.  53\% were about the Spring Security framework, dominated by authentication (45\%). The discussion \cite[\S4.3.1]{Mengetal2018a} of cross-site request forgery (CSRF) is especially worrying.  By default, Spring implicitly enables protection against this. But all the accepted answers to CSRF-related failures simply suggested disabling the check. There were no negative comments about this, and indeed a typical response is 
\begin{quote}
	\textit{''Adding csrf().disable() solved the issue!!! I have no idea why it was enabled by default.''}
\end{quote}
As of writing, there were no negative comments about this disabling of a vital security feature. This research was further developed by \cite{Chenetal2019a}  (and popularised in a security community in \cite{Zorz2019a}). Their first finding was:

\begin{quote}
	``{\emph{644 out of the 1,429 inspected answer posts
			(45\%) are insecure, meaning that insecure suggestions
			popularly exist on SO. Insecure answers dominate, in
			particular, the SSL/TLS category}}''\newline[355 insecure versus 150 secure, i.e. $>70$\%].
\end{quote} 

\subsubsection{Android}\label{sec:Android}
%Another recommendation for future work is a comprehensive survey equivalent to \cite{Drop2019} for Android textbooks;
Many Android textbooks do not rigorously consider cybersecurity. \cite{Fischeretal2017a} looked specifically at the use of resources from Stack Overflow in Android applications. The key finding was:

\begin{quote}
	``{\emph{We found that 15.4\% of all 1.3 million Android applications
			contained security-related code snippets from
			Stack Overflow. Out of these 97.9\% contain at least one
			insecure code snippet.}}''
\end{quote}

Two caveats (in opposite directions) should be noted. The labelling was conservative, in that snippets were only labelled as insecure if that was demonstrable, and, for example, mere use of outdated SSL/TLS was not automatically deemed insecure. On the other hand, the insecure snippet might have been used in a way that did not expose the insecurity. The uncritical reading of Stack Overflow was also noted in \cite[Slide 29]{Votipkaetal2019a}. Their key recommendation~\cite[Slide 32]{Votipkaetal2019a} was ``{\emph{Improve documentation: Clarify what you can(not) copy/paste}}''. 

\subsubsection{Agile}
%A further recommendation for future work is a comprehensive survey equivalent to \cite{Drop2019} for ``Agile'' textbooks. 
Many Agile textbooks have little consideration of cybersecurity. Many authors have found disconnects between Agile practices and secure software development: notably \cite{Bartsch2011a} for small projects and \cite{vanderHeijden:2018:EPS:3239235.3267426} for large projects. Agile's preference for functionality over non-functional requirements is clearly displayed in practice. \cite{Naiakshinaetal2017a} asked 20 student developers to imagine they were part of a team working on creating a social networking site for our university and to implement a password storage mechanism for this. 10 (``primed'') were explicitly told that the storage had to be secure and 10 (``unprimed'') were not. None of the unprimed ones implemented any security. 
%This would also reaffirm the point raised in section~\ref{sec:PCIDSS}, regarding the effective use of industrial case studies and authentic assessment approaches.

\subsubsection{Informal Tutorials}  % JHD changed from Resources, as these are already considered
\begin{comment}
took out foot notes for space
The web abounds with informal resources, such as tutorials as well as code snippets. How good are these, and how good are people at using these? This has been looked at by \cite{Unruhetal2017a}, taking the top five search results from Google for six queries. Of these 30 tutorials, six had SQL injection weaknesses, and three had Cross-Site Scripting\footnote{Number 7 in OWASP's Top Ten \cite{OWASP2017a}.} weaknesses. Searching for these fragments in PHP projects on GitHub found 820 instances of these fragments, of which 117 were verified manually to be vulnerable --- 80\% of which were vulnerable to SQL injection. Some students clearly make use of these resources; thus a recommendation of future work is to explore and evaluate students' (and indeed others') use of such informal resources. 
\end{comment}

The web abounds with informal resources, such as tutorials and code snippets. How good are these, and how good are people at using these? This has been looked at by \cite{Unruhetal2017a}, taking the top five search results from Google for six queries. Of these 30 tutorials, six had SQL injection weaknesses, and three had Cross-Site Scripting weaknesses. Searching for these fragments in PHP projects on GitHub found 820 instances of these fragments, of which 117 were verified manually to be vulnerable --- 80\% of which were vulnerable to SQL injection. Some students clearly make use of these resources; thus a recommendation of future work is to explore and evaluate students' (and indeed others') use of such informal resources. 

\subsection{Are the right skills and infrastructure available?}
It is well known that cybersecurity skills are in short supply, in both industry~\cite{Ackerman2019a} and academia~\cite{schneider2013}. The demand for cybersecurity skills in industry makes it difficult for academia to attract academics with knowledge, practical experience, research background and academic aspirations. As universities expand their cybersecurity provision it is not uncommon to find multiple jobs advertised at the same time. Recent example have included a professor of cybersecurity, two senior academic positions and two junior academic positions in one advert. There are other examples in the UK of cybersecurity lecturing jobs remaining unfilled for longer than a year; there are also examples of cybersecurity research groups moving en masse from one university to another.

\begin{comment}
For example, research into the state of IT conducted annually by Enterprise Strategy Group (ESG) has revealed that the skills gap in information security continues to widen and has doubled in the past five years; in 2014, 23\% of respondents to the survey stated that their organisation had a problematic shortage of information security skills -- this had climbed to 51\% at the beginning of 2018~\cite{ESG:2018}. Clearly, cybersecurity is an issue which is being felt across many industries and organisations, and is a concern which extends beyond IT leadership into the boardroom~\cite{Ackerman2019a}.


The ESG survey is international, but ESG have confirmed that the UK figures are very similar. In the UK, there has been a resurgence of job adverts to recruit academic staff with specialisms in cybersecurity over the past three years. 
\end{comment}



%\subsection{Is the right specialist equipment available?}
Delivering a practical take upon cybersecurity often requires specialist computing resources. Commonly such a laboratory will be not directly connected to the Janet network in order not to breach the operating conditions of the network. This creates further challenges in the form of acquisition and maintenance of specialist laboratory provision.

\section {Does it work?}	
%How do you know? Give some evidence of effectiveness in context.
The UK situation appears relatively advanced compared to other jurisdictions. 61\% of UK courses offer mandatory cybersecurity content, and this research was based on web scraping~\cite[Table 1]{Ruiz2019a}. As such it represents a lower bound since not all coverage will necessarily be clearly articulated in publicly available documentation online.

BCS have reported good progress in the mandating of the inclusion of cyber security within the programmes the body accredits. \cite{Cricketal2019} reported progress up to  autumn 2018. To provide an update from the start of the Autumn 2015 term, up to and including the Summer 2019 term, the BCS has carried out 82 accreditation visits including five international visits (2 in South Africa and 1 in Brunei, Cyprus, and Ireland) . The BCS identified action was required to address concerns related to cybersecurity at 23 institutions; thus, 59 institutions were already delivering cybersecurity in line with the BCS expectations.

Long-term actions (`{\emph{At Threshold }}' judgements) were expected from 14 institutions (six in 2015/16, three in 2016/17 and five in 2018/19). 13 of these judgments were across all programmes; one was specifically against a generalist masters programme only. This indicates that the BCS will expect adjustments to have taken place before the next accreditation visit. It was commonly the case that adjustments had been made to design of the programmes of study, however, the adjusted programme had not yet been delivered so the evidence base was incomplete in terms of how cybersecurity was assessed.

Short term actions were required from nine institutions; the outcomes of these actions were as follows: ({\emph{i}}) of the eleven UG programmes involved all were approved `{\emph{At Threshold}}'; ({\emph{ii}}) of the nine UG programmes involved, eight were approved and one refused; ({\emph{iii}}) of the five UG programmes involved, all were approved `{\emph{At Threshold}}'; and ({\emph{iv}}) of the three UG programmes involved, all were refused; and ({\emph{v}}) a further five which at the time of writing the outcome is not known

% \begin{itemize}                                                                
% \item Of the 11 UG programmes involved all were approved `{\emph{At Threshold}}';
% \item Of the 9 UG programmes involved, 8 were approved and 1 refused;
% \item Of the 5 UG programmes involved, all approved `{\emph{At Threshold}}';
% \item Of the 3 UG programmes involved, all 3 were refused.
% \end{itemize}

Good practice was identified at three universities by the commendation:

\begin{quote}
	``{\emph{The second-year project provides an opportunity for exploring security aspects in depth with an industrial use case.}}''
\end{quote}
\begin{quote}
	``{\emph{Hacktivity and related learning and teaching approaches}}''
\end{quote}
\begin{quote}
	``{\emph{Cyber Security Centre which permeates both the course and supports external links and opportunities for students.}}''
\end{quote}

In summary, this shows that many accredited institutions have now embedded cybersecurity in their provision, a number are in the process of doing so and a minority have chosen not to. This suggests that in the UK inclusion of cybersecurity within computer science and related degrees is becoming the norm. 

\section {Who else has done this?}	
%Where did you get the idea from? (If from published reports, please include references). How did you find out about it? Was it easy/hard to adopt? What did you change?
In jurisdictions other than the UK, PSRB's have also adjusted their expectations to enhance the cybersecurity provision. For example in the United States of America (USA), the Association of Computing Machinery (ACM) has equally had cybersecurity (IAS: ``Information Assurance and Security'') in the curriculum since 2013~\cite{ACM2013a}, but it is not the accrediting body. The Accreditation Board for Engineering and Technology (ABET) is, and is requiring IAS with effect from the 2019-20 cycle (self-study reports due 1 July 2019): more precisely \cite[Table 3]{Oudshoornetal2018a} ``{\emph{The computing topics must include: \dots{} Principles and practices for secure computing\dots}}''. This should mean that  all accredited universities should be compliant by 2025 (due to their six-year cycle).However the challenges related to pedagogy, underpinning educational resources, skills shortages and resource requirements are a global challenge.

\begin{comment}


Within the computer science education research community there is a growing appreciate for the need to enhance underpinning resources for example \cite{Drop2019}. Also educational cybersecurity is an increasing common theme at education research conferences.

%is this enough?

\end{comment}

\begin{comment}
% is this nececessary?

The ACM/IEEE-CS Joint Task Force on Computing Curricula~\cite[p.~97]{ACM2013a} takes a distinct view on the Knowledge Areas (KAs):

\begin{quote}
	``{\emph{The Information Assurance and Security KA is unique among the set of KAs presented here
			given the manner in which the topics are pervasive throughout other KAs.}}''
\end{quote}

It proposes nine ``core'' hours and 63.5 distributed across the other KAs. Nevertheless, the situation on the ground in the USA is different~\cite{Ackerman2019a}:

\begin{quote}
	``{\emph{Universities suffer shortcomings, as well. Roughly 85 of them offer undergraduate and/or graduate degrees in cybersecurity. There is a big catch, however. Far more diversified computer science programs, which attract substantially more students, don't mandate even one cybersecurity course.}}''
\end{quote}

% delete to here maybe?
\end{comment}



\section {What will you do next?}	
As indicated in the previous section there are a number of challenges
that have not been fully addressed:

\begin{enumerate}
	\item There is a need for research related to the effectiveness of alternative pedagogies for the delivery of cybersecurity
	\item Many of the common underpinning resources (textbooks or informal resources) do not address cybersecurity satisfactorily. This is a development opportunity for the computer science education community to address this;
	\item There is a skills gap. The computer science education community could help address this by growing the number of appropriate qualified potential lecturers;
	\item The effectiveness of alternative resources to support the provision of cyber security is another area that could benefit from further research.
\end{enumerate}

\begin{comment}
It is the view of the authors that the computer science education community collectively should address these (and other emergent) issues related to the delivery of good quality embedded cybersecurity education.
\end{comment}

\section{Why are you telling us this?}	
%TO DO What is interesting, or useful, about this to someone else?
% Examples of security breaches are common feature of the news. One
% eye-catching example is British Airways is facing a \pounds183 million
% fine for last year's breach of its security systems
% \cite{BBC2019}. Incidents like this highlight the importance of
% cybersecurity generally. 

Most institutions in the UK now include aspects of cybersecurity
in their general undergraduate computer science provision in the same manner that
they include legal, social, ethical and professional issues; it is now time to
consider how to further enhance the quality of provision, with a focus
on pedagogy, assessment and progression. How can cybersecurity
receive the same level of attention in terms of pedagogic research and
practice as say, programming or CS1? We feel that there is a significant opportunity for
the UK computer science academic community, in collaboration with a
range of key stakeholders, to drive forward this new educational
research priority.

\section{Acknowledgments}
The authors wish to thank Sally Pearce, Academic Accreditation Manager at BCS, The Chartered Institute for IT for supplying the summary information related to accreditation of UK degree programmes. Many people, accreditors and accredited, have contributed to accreditation practice in the UK (and elsewhere), and spreading good practice.  All authors' institutions are members of the Institute of Coding, an initiative funded by the Office for Students (England) and the Higher Education Funding Council for Wales.
%%
%% The next two lines define the bibliography style to be used, and
%% the bibliography file.
\bibliographystyle{ACM-Reference-Format}
%\bibliography{sample-base}
\bibliography{CEP2020}


\end{document}
\endinput
%%
%% End of file `sample-sigconf.tex'.
