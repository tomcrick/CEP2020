%%
%% This is file `sample-sigconf.tex',
%% generated with the docstrip utility.
%%
%% The original source files were:
%%
%% samples.dtx  (with options: `sigconf')
%% 
%% IMPORTANT NOTICE:
%% 
%% For the copyright see the source file.
%% 
%% Any modified versions of this file must be renamed
%% with new filenames distinct from sample-sigconf.tex.
%% 
%% For distribution of the original source see the terms
%% for copying and modification in the file samples.dtx.
%% 
%% This generated file may be distributed as long as the
%% original source files, as listed above, are part of the
%% same distribution. (The sources need not necessarily be
%% in the same archive or directory.)
%%
%% The first command in your LaTeX source must be the \documentclass command.
\documentclass[sigconf]{acmart}

%%
%% \BibTeX command to typeset BibTeX logo in the docs
\AtBeginDocument{%
  \providecommand\BibTeX{{%
    \normalfont B\kern-0.5em{\scshape i\kern-0.25em b}\kern-0.8em\TeX}}}

%% Rights management information.  This information is sent to you
%% when you complete the rights form.  These commands have SAMPLE
%% values in them; it is your responsibility as an author to replace
%% the commands and values with those provided to you when you
%% complete the rights form.
\setcopyright{acmcopyright}
\copyrightyear{2020}
\acmYear{2020}
\acmDOI{10.1145/1122445.1122456}

%% These commands are for a PROCEEDINGS abstract or paper.
\acmConference[CEP '20]{CEP '20: ACM Computing Education Practice}{January 9, 2020}{Durham, UK}
\acmBooktitle{CEP '20: Proceedings of the 3rd Conference on Computing Education Practice,
  June 9, 2020, Durham, UK}
\acmPrice{15.00}
\acmISBN{978-1-4503-9999-9/18/06}

\usepackage{dirtytalk}

%%
%% Submission ID.
%% Use this when submitting an article to a sponsored event. You'll
%% receive a unique submission ID from the organizers
%% of the event, and this ID should be used as the parameter to this command.
%%\acmSubmissionID{123-A56-BU3}

%%
%% The majority of ACM publications use numbered citations and
%% references.  The command \citestyle{authoryear} switches to the
%% "author year" style.
%%
%% If you are preparing content for an event
%% sponsored by ACM SIGGRAPH, you must use the "author year" style of
%% citations and references.
%% Uncommenting
%% the next command will enable that style.
%%\citestyle{acmauthoryear}

%%
%% end of the preamble, start of the body of the document source.
\begin{document}

%%
%% The "title" command has an optional parameter,
%% allowing the author to define a "short title" to be used in page headers.
\title{Accreditation: Post Shadbolt}

%%
%% The "author" command and its associated commands are used to define
%% the authors and their affiliations.
%% Of note is the shared affiliation of the first two authors, and the
%% "authornote" and "authornotemark" commands
%% used to denote shared contribution to the research.


%\begin{comment}

\author{Tom Crick}
\affiliation{%
  \institution{Swansea University}
  \city{Swansea}
  \country{UK}
}
\email{thomas.crick@swansea.ac.uk}


\author{James H. Davenport}
\affiliation{%
  \institution{ University of Bath}
  \city{Bath}
  \country{UK}
}
\email{j.h.davenport@bath.ac.uk}

\author{Alastair Irons}
\affiliation{%
  \institution{ Sunderland University}
  \city{Sunderland}
  \country{UK}
}
\email{alastair.irons@sunderland.ac.uk}

\author{Tom Prickett}
\affiliation{%
  \institution{ Northumbria University}
  \city{Newcastle upon Tyne}
  \country{UK}
}
\email{tom.prickett@northumbria.ac.uk}
%\end{comment}


%%
%% By default, the full list of authors will be used in the page
%% headers. Often, this list is too long, and will overlap
%% other information printed in the page headers. This command allows
%% the author to define a more concise list
%% of authors' names for this purpose.
\renewcommand{\shortauthors}{Crick, Davenport,  Irons, and Prickett.}

%%
%% The abstract is a short summary of the work to be presented in the
%% article.
\begin{abstract}
The promotion of quality via Accreditation by Professional, Statutory and Regulatory Bodies (PSRBs) is very much a feature of higher education in the United Kingdom (UK).  In %this  %%JHD: "this" didn't refer to anything
a changing  environment, there is a need for accreditation regimes to evolve in order to maximise the value they provide to Higher Education Institutions (HEIs).  This paper provides a summary in the context of one professional body (BCS, The Chartered Institute For IT) of
what has happened in response to the recommendations of the Shadbolt review,  ongoing enhancement projects and the related future conversations that will help steer further enhancements.
\end{abstract}

%%
%% The code below is generated by the tool at http://dl.acm.org/ccs.cfm.
%% Please copy and paste the code instead of the example below.
%%
\begin{CCSXML}
<ccs2012>
<concept>
<concept_id>10003456.10003457.10003527.10003529</concept_id>
<concept_desc>Social and professional topics~Accreditation</concept_desc>
<concept_significance>500</concept_significance>
</concept>
</ccs2012>
\end{CCSXML}

\ccsdesc[500]{Social and professional topics~Accreditation}

%%
%% Keywords. The author(s) should pick words that accurately describe
%% the work being presented. Separate the keywords with commas.
\keywords{Accreditation, Computer Science Education, Curricula Design}


%%
%% This command processes the author and affiliation and title
%% information and builds the first part of the formatted document.
\maketitle
\section {What is it?}
%A short description of the practice you're presenting
The Shadbolt Review \cite{shadbolt2016shadbolt} investigated the relatively high unemployment rates graduates of computer science and related degrees in the UK and the role of degree accreditation in promoting employability. The Shadbolt review noted that 
% Next line JHD: I think we need to keep saying this!
the disparity in `raw' unemployment rates was largely accounted for by prior achievement and socio-economic factors, and that
rates of graduate unemployment were declining and varied considerably between geographic location and type of HEI. The discipline was already acting proactively to address the issue.  Related professional bodies have enhanced their accreditation processes in response these challenges. BCS has made significant adjustments to their accreditation processes since 2015. These were partly in response to the discussions related to and the recommendations of Shadbolt but also to other changes in the sector and the discipline.  These have been communicated to the assessor community and to those responsible for leading the development of HEI accreditation applications. However, there are many academics who are less intimately involved in accreditation: this paper serves to communicate these and future enhancements to them and the wider computer science education community.
\section {Why are you doing it?}	
%What happened before? What is it changing / replacing / improving? What gap is it filling?
The Shadbolt review \cite[p.~8]{shadbolt2016shadbolt} makes two main recommendations for the enhancement of accreditation regimes, one related to the Academic Accreditation of Degree Courses and the other related to Engaging Industry in Accreditation. Criticism of the accreditation of degree programmes is not new. There is a history of claiming the processes are unnecessarily bureaucratic and constrain  innovation \cite{Harvey2004},  and there are dangers of accreditation streams being revenues streams in their own right rather than for the benefit of a discipline or wider society \cite{Knight_2015}. Equally its value has been highlighted particularly in the context of a potentially globally mobile workforce \cite{Knight_2015}. Hence, the challenge set for BCS and other accreditation providers that operate in the Computer Science discipline is broadly to: increase awareness and value of accreditation; focus upon outputs; maintain internationally recognised standards;  respond to emerging technology trends and developments; promote enhancement and innovation; engage industry; and reduce the perceived bureaucracy involved.

\begin{comment}

%%\begin{quote} 
	"Recommendation 9 - Academic Accreditation of Degree Courses.
	
	BCS, IET and Tech Partnership should ensure that existing systems of degree course
	accreditation are flexible, agile, and enable HE providers to respond to changing
	demand and emerging technological trends and developments. Accreditation of courses
	should be focused on outputs. Accrediting bodies should work to increase awareness
	and value of accreditation so that it is valued by HE providers, students and employers,
	and consider how their role can provide a forum for engagement between HE and
	employers.

	Recommendation 10 - Engaging industry in accreditation.

	Employers, through employer groups, such as Tech Partnership, should engage more
	consistently with HE providers and BCS \& IET to ensure accreditation is effective and
	reflects current industry demand."
%%\end{quote}
\end{comment} 


\section {Where does it fit?	}
%A short description of your teaching context. You may, for instance, include a description of intake, class size, curriculum sequence; anything that's necessary for others to understand your situation. How do things work at your institution?

In the United Kingdom the most common form of accreditation in the Computer Science and related discipline areas is by professional bodies.  BCS, The Chartered Institute for IT (BCS) and the Institution of Engineering and Technology (IET) are the main bodies involved. The accreditation provided by these %institutes 
are underpinned by international initiatives such as the Washington Accord \cite[for CEng]{Washington2019} and Seoul Accord \cite[for CITP]{Seoul2019}. These memoranda support the internationalising of the curriculum and promote consistency and parity in Computer Science education globally. 

In accreditation IET and BCS broadly check two things \cite{BCS2018a, IET2019}. 

\begin{enumerate}
    %\item Is the quality of experience provided by a HEI appropriate for accreditation?
    \item Are the exit standards of the programme appropriate to support accreditation?
     A number of standards are considered including entry, progression, retention, awards and graduate employability.  This is supplemented by other evidence of the quality of the provision, for example external examiners reports, the most recent subject review, annual review information, evidence of employer involvement, linkage with research. Together this evidences that a programme is of an appropriate quality to support accreditation.
    \item Are the curricula exit standards of a programme consistent with the learning outcomes expected for the accreditation sought? The expected exit standards should conform with the international memorandum (Washington or Seoul Accord or both). 
\end{enumerate}

BCS is addressing the recommendations of the Shadbolt Review in the following ways.

\subsection{The value of accreditation}
What is the value of an accredited degree?  BCS is in the process of initiating a conversation to explore the value proposition of accreditation from the point of view of other stakeholders. The following are aspects of the value.
%\subsubsection{Maintaining standards}
\subsubsection{Raising output standards}
\begin{comment}
version 1
BCS refuses accreditation for programmes that do not meet the expected output standards or when achievement of the curricula exit standards for the accreditation sought are not met. "\% of graduates in related professional 6 months after graduating" \cite[p8]{BCS2019b} is the employability output standard applied to each programme. Departments are also requested to  "A 1.1.6 Describe how employability skills are developed within the students and how students are supported in their professional developmesion nt" \cite [p3]{BCS2019b}. Both these criteria were introduced in response to Shadbolt.
\end{comment}
\begin{comment}
version 2
BCS refuses accreditation for programmes that do not meet the expected output standards. "\% of graduates in related professional 6 months after graduating" \cite[p8]{BCS2019b} is the employability output standard applied to each programme. Departments are also requested to  "A 1.1.6 Describe how employability skills are developed within the students and how students are supported in their professional development" \cite [p3]{BCS2019b}. Both these criteria were introduced in response to Shadbolt.
\end{comment}
BCS can and has refused accreditation for programmes that are not of an appropriate quality for accreditation. However, an enhancement-oriented rather than prescriptive approach is adopted. Quality is considered on a holistic basis. In response to the Shadbolt two criteria were introduced. These are "\% of graduates in related professional 6 months after graduating"  \cite [p8]{BCS2019b} and "A 1.1.6 Describe how employability skills are developed within the students and how students are supported in their professional development" \cite [p3]{BCS2019b}.
\subsubsection{Promoting internationally agreed standards}
The BCS accredits to internationally agreed standards, which evidence the global parity of the degree programmes accredited. BCS refuses accreditation when achievement of the curricula exit standards for the accreditation sought are not met.  This assists in the global mobility for the graduates of accredited programmes. Part of the value here is linked to the value of professional registration. BCS currently has a project investigating how to enhance the value of professional registration to students, graduates and early career professionals.
\subsubsection{Ensuring curricula relevance}
With the agreement of the sector (normally reached through the Council of Professors and Heads of Computing (CPHC)), it is possible to agree and promote curricula change. The most recent example of this is the inclusion of cybersecurity in all accredited degrees \cite{Cricketal2019}. 
%BCS is in the process of mandating %% JHD strengthens text
BCS mandates the inclusion of security in all accredited degree programmes to a standard agreed between professional bodies, industry and government. %This is a work in progress   %% JHD removed this as it seems negative
BCS has been requiring coverage since 2015 \cite[p.~17--18]{BCS2018a} with the result that all accredited universities must be compliant by 2020 (due to the five-year cycle of accreditation) \cite{Cricketal2019}. 

This year's Royal Institution Christmas lecture considers ethical questions related to computing and mathematics \cite{RoyalInstitution2019}. This is a curricula element that has been mandated by BCS accreditation in the form of legal, social, ethical and professional issues for a number of years \cite{Brooke2018}.

Group working experience is mandated by BCS (criterion 2.3.1 \cite [p31] {BCS2018a}). This is extremely valued by employers, but typically disliked by students at the time, and features negatively in many student comments. Hence if UK universities, which are heavily judged by current student opinion, were left to themselves without accreditation, they would probably remove group working.  % JHD strengthened 'may' to 'would probably'

\subsubsection{Disseminating good practice}
BCS accreditation panels have been identifying aspects of good practice for wider dissemination for a number of years. 
%By commending good practices and promoting these to the wider community. %%JHD: orphan clause?
Since 2016, the process has been refined to more clearly signpost commendations. BCS is currently in the process of refining the promotion of good practice. One short term initiative, from autumn 2019, visited HEIs will be offered the opportunity to showcase an aspect of their provision or practice. The intention is to generate further good practice examples that a visiting panel can potentially commend.
\subsubsection{Industry relevance}
All BCS accreditation panels include an industrial assessor whose role includes ensuring that programmes are providing high quality, up to date and relevant material that produce graduates who are able to enter a competitive employment market. Evidence that a visited HEIs' mechanisms for engaging with industry are supporting graduates to evidence they are achieving the expected exit standards are sought from a variety of sources including the curricula studied; the assessments and examinations sat; and the engagement mechanisms themselves. 
\subsubsection{Independent peer review}
Peer review is commonly used in quality regimes in higher education and in periodic review processes. In most cases the HEI being reviewed chooses the peer. As part of a BCS (or other professional body) accreditation a HEI has less choice regarding who completes a review and hence such a review is arguably more truly independent. BCS review panels contain a minimum of two experienced assessors who have strong awareness of the discipline norms across the UK sector.

\subsection{Accrediting work experience}
BCS has introduced accreditation to Professional Registration for IT Technicians (RITTech) as a mechanism for accrediting industrial experience gained during a placement, foundation degree or work-experience as part of a degree apprenticeship \cite[p8]{BCS2018a}. This product development was introduced in response to Shadbolt.

\subsection{Driving improvement}
A number of data sources have been employed to help drive the improvement efforts. The Shadbolt report led to a number of enhancements. BCS Secretariat continually evaluate via the use of opinion surveys and informal conversations. Views are gathered related to briefing sessions, pre-visit communications and visits. Following a visit BCS panels engage in peer review. The feedback gained from these sources is explored and opportunities for enhancement agreed at accreditation committee meetings. Prioritised working groups then complete the enhancement projects. 

\subsection{Reducing bureaucracy/enhancing practices}
A review of BCS Accreditation practice has been taking place since 2015 with the intention of adopting an enhancement-oriented agenda; reducing the amount of bespoke documentation that is required; using technology to assist the process, whilst enforcing the international standards. Working through the process from a visited HEIs prospective the review results in the following changes.

\subsubsection{HEI briefing}
For a number of years all visited HEIs have been encouraged to attend a briefing to explore the BCS requirements with a focus upon any changes since the last visit and common challenges. % that present themselves as part of visits. 
Attendance was understandably mixed, partly due to travel. Since 2017 all Briefings have taken place by video conference. Feedback upon this approach has been very positive.

\subsubsection{The application itself}
From a paper and memory stick / CD submission before 2015, BCS has moved to a fully electronic submission. BCS is flexible in how the information is provided, a range of secure cloud-based file sharing systems have been employed, submission via the use of Virtual Learning Environment has been completed and a minority of institutions have opted to create a website related to the submission. Most of the application consists of an evidence base which HEIs will already have. This is supplemented by a summary of the provision and each programme in which the department are welcome to reference existing resources. HEIs are still required to provide a mapping of where the BCS requirements are taught and assessed within each programme. This is required to evidence where the accreditation-specific requirements (such as security or ethics) are met in a particular programme.

\subsubsection{Areas for discussion at visits}
It is now normal practice for a BCS panel to communicate likely areas of discussion to a visited HEI prior the visit wherever  possible (but HEI applications may be late or other issues can emerge as part of the visit).   This has resulted in discussions tending to become more collegiate and supportive. 
\begin{comment}
\subsection{Promoting the value of accreditation}
The value proposition discussed in this document is a definition of the value of accreditation from those who are intimately involved. BCS is in the process of initiating a conversation to explore the value proposition of accreditation from the point of view of other stakeholders.
\end{comment}

\section {Does it work?}
% how do you know? Give some evidence of effectiveness in context.

HEIs choose which programmes are submitted or not in an accreditation application. As such only a sample of provision may be considered. However, a proxy for success is the anonymous feedback BCS obtains post visit. Some key aspects of this feedback are shown in Table 1 ~\ref{table:1}. Overall the results are positive, indicating the process is generally valued and shows the increased attendance of the virtual briefing over the physical one.

% Please add the following required packages to your document preamble:
% \usepackage{graphicx}
%\begin{table}[]
\begin{table}[h!]
  \caption{Selected results from accreditation visit feedback survey}
  \label{table:1}
%\resizebox{\textwidth}{!}{%
%\begin{tabular}{lllll}
\begin{tabular}{ | p{5cm}|p{.75cm}|p{.75cm} |p{.75cm} |}
\hline
 & 2017 & 2018 & 2019   \\ \hline
Responses & 12 & 14 & 16   \\
\hline
Attended pre-visit briefing & 6 & 14 & 16   \\
\hline
The visit felt to be worthwhile: & & &  \\
Strongly Agree &92\% & 93\% &88\% \\
Agree &0\% & 7\% &0\% \\
Tend to Agree &0\% & 0\% &6\% \\
Strongly Disagree &8\% & 0\% &0\% \\
No reply &0\% & 0\% &6\% \\
\hline
Overall how satisfied with the visit? & & &  \\
Very Satisfied &83\% & 86\% &94\% \\
Quite Satisfied &8\% & 14\% &6\% \\
Not Satisfied at all &8\% & 0\% &0\% \\
\hline

\end{tabular}%
%}
\end{table}
\cite{Cricketal2019} shows that accreditation has driven Cybersecurity in the UK, as opposed to the US, where is it also in the recommended curriculum, but the correspinding accreditation requirements are much more recent.

\section {Who else has done this?}	
%Where did you get the idea from? (If from published reports, please include references). How did you find out about it? Was it easy/hard to adopt? What did you change?

In the UK for the broad computing area, in addition to IET and BCS providing accreditation a number of agencies provide endorsement. These schemes are commonly intended to promote employabililty of graduates. This bodies include Tech Partnership Degrees ~\cite{TP2019}; TIGA a trade association representing the UK's games industry and  Screenskills (formerly Creative Skillset) \cite{Screenskills2019}; The Chartered Society of Forensic Sciences \cite{CSOFS2019}; and the National Cyber Security Centre (NCSC) \cite{NCSC2018a}. There is little published regarding the effectiveness or otherwise of these endorsements.

\begin{comment}

In addition to IET and BCS, accreditation in the broad computing area in the UK is also being performed by a few different agencies. A number of bodies are providing endorsements intended to promote employability. Tech Partnership Degrees provides endorsements to Higher Education programmes with specific curricula elements aimed at job market requirements \cite{TP2019}. Tech Partnership Degrees have a specialist scope, endorsing programmes in the area of IT Management for Business and Software Engineering for Business. For games related degrees, TIGA a trade association representing the UK's games industry and  Screenskills (formerly Creative Skillset) \cite{Screenskills2019} operate a similar role. For digital forensics The Chartered Society of Forensic Sciences \cite{CSOFS2019} provides industry related endorsements.  National Cyber Security Centre (NCSC) is a UK Government organisation tasked with enhancing the cybersecurity of the UK and acredits programmes with significant cybersecurity content \cite{NCSC2018a}. There is little published regarding the effectiveness or otherwise of these endorsements.

\end{comment}

The Institute of Coding (IoC) is a not for profit organisation that intends to enhance how Digital Skills are developed in Higher Education in the UK \cite{Davenportetal2019a}. A micro-credentialing approach is being taken to its proposed accreditation regime. This could potentially augment the current recognised pathways to professional accreditation by providing a more fine-grained alternatives that could be useful to some employers or employees who wish to evidence their achievements in an accredited manner.  One of the challenges in this work is the lack of an agreed standard (in the way there is for the Seoul and Washington Accords) for micro-credentialing. The BCS is actively working with IoC and intends to collaborate with the IoC in any initiatives of mutual benefit.

The wider Computer Science discipline has also responded actively to the challenges presented by Shadbolt with CPHC operating four working groups \cite{cphc_2016}.  The Royal Academy of Engineering Visiting Industrial Professor Scheme \cite{royal} has promoted further engagement between industry and academia.

\section {What will you do next?}	
%Will you vary this, or develop it further?
As discussed earlier,  BCS % is in the process of initiating %%JHD very weak
has initiated
a conversation to explore the value proposition of accreditation from other stakeholders points of view. The wider BCS is exploring the value proposition BCS membership represents to students, graduates and early career professionals. As indicated previously an industrial assessor is a critical part of every visit panel. There are also interesting developments being led by the IoC related to micro-credentialing. BCS will continue to consider opportunities for the further involvement of employers in accreditation. BCS is currently in the process of refining its processes for the promotion of good practice. In addition to these enhancements, there are number of enhancements either in progress or planned for the future which are considered next.

%\subsection{Internationalization}
BCS accreditation primarily takes place in the UK, however there are a small but growing number of institutions outside the UK that BCS accredits. Under the terms of the Washington and Seoul Accords, BCS does not engage in accreditation in the jurisdictions of Accord signatories without first consulting with the related local professional body.  On occasions BCS guidelines employ English / Welsh higher education terms without indicating that local equivalents are equally acceptable. The next set of guidelines will address this shortcoming.

%\subsection {Engineering Council and Accreditation of Higher Education Programmes (AHEP) Version 4}
In the British jurisdiction of the Washington accord, the approach adopted is unusual in that the Engineering Council extends the license to accredit \emph{Chartered Engineer} to a large number of professional bodies \cite{EC2019}. This is not the practice adopted in other jurisdictions, for example in the United States of America ABET is the sole body. A recent audit by the Washington Accord has highlighted divergence in practice in some areas with respect to Chartered Engineer Accreditation. The Engineering Council is in the process of seeking further consistency. This has led to the new rules with respect of Compensation and Condonement for Eng accreditation \cite{EC2018}. Further changes are embedded in Accreditation of Higher Education Programmes version 4 (AHEP 4). The BCS requirements have been updated to reflect the expectations of Compensation and Condonement. Once AHEP 4 is finalised further work will be required to address these updates. Among other proposals, AHEP 4 intends to extend Chartered Engineering accreditation to include diversity and widening participation data (of staff and students) as part of the metrics that assess the quality of provision. The Computer Science discipline could clearly do better in this regard, so this should be a positive inclusion.

%\subsection{Agility of processes and changing external environment}
There are considerable external pressures placed upon HEIs, by for example the Research Excellence Framework (REF) and the Teaching Excellence Framework (TEF), which could reduce the priority placed upon accreditation as a mechanism for supporting enhancement. BCS continues to review the bespoke documentation it requires and to make reductions when possible. Care is taken in visit reports to ensure positive aspects of provision are emphasised. This is in part intended to provide evidence that could be used to support future TEF related submissions. Discussions regarding the agility of processes are incorporated into two annualBCS Accreditation Committee meetings to identify enhancements. 

%\subsection {Improve communications and support}
As part of its operation BCS continually reviews and monitors the support process it employs for HEIs seeking accreditation and hence will continue to enhance processes in this area.
There is a strong and vibrant community of academics and industrialists who actively participate in accreditation. However, there are many involved with whom communication could benefit from being more active. A number of initiatives are in process to enhance these communications. For HEIs there are ongoing discussion within BCS regarding establishing a periodic bulletin which could be used to share updates of accreditation practice / procedure, highlight HEIs having their first visit, sharing good practice examples, promote assessor recruitment and so on. Assessor Peer Review has indicated more regular communication with assessors could be of benefit. Virtual link up sessions for assessors are proposed. The BCS are in the process of contacting assessors to determine likely interest and thoughts related to relevant topics for such link ups.

\section{Why are you telling us this?}	
%What is interesting, or useful, about this to someone else?
One of the recommendations of Shadbolt is the value of accreditation should be more clearly communicated to stakeholders. This paper is part of a set of initiatives to achieve precisely that.

Much has changed in BCS accreditation in the last few years. For many academics their only experience of BCS accreditation is the quinquennial BCS accreditation visit. Accreditation by the BCS is performed by panels of BCS assessors. An assessor may be an industrialist or an academic. The size of a panel varies depending upon the number of programmes a visited HEI is putting forward for consideration. A panel will always include an industrial assessor and two or more academic assessors.  HEIs will have at most two academic assessors from their staff.  There is a large pool of academic assessors and as such not all assessors will complete an accreditation visit every year. Conversely, not all HEIs have an academic assessor. Hence a significant set of the Computer Science education community are not informed regarding the evolution of accreditation practice. As such the intention of the paper is share to the wider community the developments made, the future enhancement aspirations and the forthcoming opportunities to engage in further conversations.

\section{Acknowledgments}
The authors wish to thank Sally Pearce, Academic Accreditation Manager at BCS, The Chartered Institute for IT for supplying the summary information related to accreditation of UK degree programmes. Many people, accreditors and accredited, have contributed to accreditation practice in the UK (and elsewhere), and spreading good practice.  All authors' institutions are members of the Institute of Coding, an initiative funded by the Office for Students (England) and the Higher Education Funding Council for Wales.
%%
%% The next two lines define the bibliography style to be used, and
%% the bibliography file.
\bibliographystyle{ACM-Reference-Format}
%\bibliography{sample-base}
\bibliography{CEP2020}


\end{document}
\endinput
%%
%% End of file `sample-sigconf.tex'.
